%
% Draft  document voiddocker.tex
%
 
\documentclass{article}  % Latex2e
\usepackage{graphicx,lscape,subfigure}
 

\title{Installing docker in Void Linux}
\author{Neville Jackson}
\date{30 June 2022} 

\begin{document} 

\maketitle      

\section{Introduction} 
A reasonable explanation of what docker is and how it works is given in this tutorial~\cite{dock:00}. In one sentence docker is like a VM , but it only containerizes a selected part of user space. It was originally intended to be used for providing an isolated environment for software development. Other uses have developed. 

Two key concepts are images and containers. A docker image is the software which is containerised. There is a registry of useful images called {\em Docker Hub}~\cite{dock:02}. A docker container  a copy of an image, along with some organization which  allows it to be moved, installed, modified and run.

\section{Install steps}
\label{sec:install}
Void has several docker-related packages
\begin{verbatim}
\end{verbatim}
I chose initially to install just {\em docker} and see what dependencies it dragged in
\begin{verbatim}
# xbps-install docker

Name       Action    Version           New version            Download size
docker-cli install   -                 20.10.12_1             9979KB 
runc       install   -                 1.1.3_1                3024KB 
containerd install   -                 1.5.7_1                46MB 
moby       install   -                 20.10.9_2              23MB 
tini       install   -                 0.19.0_1               299KB 
docker     install   -                 20.10.12_1             545B 

Size to download:               83MB
Size required on disk:         253MB
Space available on disk:       263GB

Do you want to continue? [Y/n] y
...........
6 downloaded, 6 installed, 0 updated, 6 configured, 0 removed.
# exit

\end{verbatim}

I then ran a test
\begin{verbatim}
# docker info
lient:
 Context:    default
 Debug Mode: false

Server:
ERROR: Cannot connect to the Docker daemon at unix:///var/run/docker.sock. Is the docker daemon running?
errors pretty printing info
# 
\end{verbatim}

 So I have to start the docker daemon - there are two of them. Unlike most distros, Void does not start daemons automatically when a package is installed. In the runit init system daemons are started by making a filesystem link
\begin{verbatim}
# ln -s /etc/sv/containerd /var/service/containerd
# ln -s /etc/sv/docker /var/service/docker
\end{verbatim}

Then test again
\begin{verbatim}
# docker info
Client:
 Context:    default
 Debug Mode: false

Server:
 Containers: 0
  Running: 0
  Paused: 0
  Stopped: 0
 Images: 0
 Server Version: 20.10.9
 Storage Driver: overlay2
  Backing Filesystem: extfs
  Supports d_type: true
  Native Overlay Diff: true
  userxattr: false
 Logging Driver: json-file
 Cgroup Driver: cgroupfs
 Cgroup Version: 1
 Plugins:
  Volume: local
  Network: bridge host ipvlan macvlan null overlay
  Log: awslogs fluentd gcplogs gelf journald json-file local logentries splunk syslog
 Swarm: inactive
 Runtimes: io.containerd.runc.v2 io.containerd.runtime.v1.linux runc
 Default Runtime: runc
 Init Binary: docker-init
 containerd version: UNSET
 runc version: 
 init version: 
 Security Options:
  seccomp
   Profile: default
 Kernel Version: 5.15.45_1
 Operating System: Void Linux
 OSType: linux
 Architecture: x86_64
 CPUs: 12
 Total Memory: 62.79GiB
 Name: trinity
 ID: 47TV:VTCL:OLKZ:4N3A:UPAV:MCEJ:CTRJ:HQOD:XVHC:CVJO:RUMU:YZOW
 Docker Root Dir: /var/lib/docker
 Debug Mode: false
 Registry: https://index.docker.io/v1/
 Labels:
 Experimental: false
 Insecure Registries:
  127.0.0.0/8
 Live Restore Enabled: false
#
\end{verbatim}
 That looks OK now . There are 0 containers running , that is correct.

Now lets run a simple test container
\begin{verbatim}
# docker run hello-world
Unable to find image 'hello-world:latest' locally
latest: Pulling from library/hello-world
2db29710123e: Pull complete 
Digest: sha256:53f1bbee2f52c39e41682ee1d388285290c5c8a76cc92b42687eecf38e0af3f0
Status: Downloaded newer image for hello-world:latest

Hello from Docker!
This message shows that your installation appears to be working correctly.

To generate this message, Docker took the following steps:
 1. The Docker client contacted the Docker daemon.
 2. The Docker daemon pulled the "hello-world" image from the Docker Hub.
    (amd64)
 3. The Docker daemon created a new container from that image which runs the
    executable that produces the output you are currently reading.
 4. The Docker daemon streamed that output to the Docker client, which sent it
    to your terminal.

To try something more ambitious, you can run an Ubuntu container with:
 $ docker run -it ubuntu bash

Share images, automate workflows, and more with a free Docker ID:
 https://hub.docker.com/

For more examples and ideas, visit:
 https://docs.docker.com/get-started/

# 
\end{verbatim} 

Well, at least it gives me some info. If I now repeat the {\em docker info} command
\begin{verbatim}
# docker info
Client:
 Context:    default
 Debug Mode: false

Server:
 Containers: 1
  Running: 0
  Paused: 0
  Stopped: 1
 Images: 1
 Server Version: 20.10.9
........
\end{verbatim}
So there is now 1 container present, it is stopped, and there is one image.
I dont see any files in my home directory , so where has docker put things? It  seems docker stores files in /var/lib/docker
\begin{verbatim}
#ls -F /var/lib/docker
buildkit/    image/    overlay2/  runtimes/  tmp/    volumes/
containers/  network/  plugins/   swarm/     trust/
# 
\end{verbatim}
All directories. The overlay2  directory contains the image of the hello-world container.

 We dont really want to keep the hello-world container so lets delete it
\begin{verbatim}
# docker container prune
WARNING! This will remove all stopped containers.
Are you sure you want to continue? [y/N] y
Deleted Containers:
b6ed07e4a0d03a6930092864e7201e3cce7740153d42a24e4f221427792368ab

Total reclaimed space: 0B
# 
\end{verbatim}
We use {\em prune} because it is a stopped container. {\em docker info} now reports 0 containers, but still 1 image? How do we remove the image?
\begin{verbatim}
# docker image ls
REPOSITORY    TAG       IMAGE ID       CREATED        SIZE
hello-world   latest    feb5d9fea6a5   9 months ago   13.3kB
# docker image rm hello-world
Untagged: hello-world:latest
Untagged: hello-world@sha256:53f1bbee2f52c39e41682ee1d388285290c5c8a76cc92b42687eecf38e0af3f0
Deleted: sha256:feb5d9fea6a5e9606aa995e879d862b825965ba48de054caab5ef356dc6b3412
Deleted: sha256:e07ee1baac5fae6a26f30cabfe54a36d3402f96afda318fe0a96cec4ca393359
# 
\end{verbatim}
{\em docker info} now reports 0 images. 

\section{Docker Desktop}
We have been using the command line interface (CLI) to docker, which is the void package {\em docker-cli}. There is a Docker Desktop available from the docker website~\cite{dock:01} as a .deb.or .rpm file.  Void cannot use .deb.or .rpm files, and there does not seem to be a Void Docker Desktop package, so we are stuck with the CLI in Void.

The Docker Desktop install webpage~\cite{dock:01} says that in addition to the .deb or .rpm package, Docker Desktop requires KVM support, and QEMO, and Gnome or KDE DTE. My Void installation has none of those installed. The KVM and QEMU packages exist
\begin{verbatim}
xbps-query -Rs KVM
[-] aqemu-0.9.4_1                       GUI to QEMU and KVM emulators, writte...
[-] barrier-2.4.0_1                     Open-source KVM software based on Syn...
[-] barrier-gui-2.4.0_1                 Open-source KVM software based on Syn...
[-] docker-machine-driver-kvm-0.10.1_1  KVM driver for docker-machine
[-] docker-machine-driver-kvm2-1.24.0_1 Minikube-maintained KVM driver for do...
[-] virtme-0.1.1_4                      Easy way to test your kernel changes .

xbps-query -Rs qemu
[-] aqemu-0.9.4_1            GUI to QEMU and KVM emulators, written in Qt4
[-] novaboot-20191023_2      Tool that automates booting of operating systems...
[-] qemu-7.0.0_1             Open Source Processor Emulator
[-] qemu-ga-7.0.0_1          QEMU Guest Agent
[-] qemu-user-static-7.0.0_1 QEMU User-mode emulators (statically compiled)
[-] qemuconf-0.2.1_3         Simple qemu launcher with config file support
[-] virtme-0.1.1_4           Easy way to test your kernel changes in qemu/kvm
\end{verbatim}
but none of these are starred, which would indicate installed. 
So {\em docker} is able to run without these requirements, but not {\em docker desktop}.

\section{An interactive container}
Lets try and run a container which we can interact with. The docker run command has options which setup an interactive shell. To see them use the --help option as follows
\begin{verbatim}
$ docker run --help

Usage:  docker run [OPTIONS] IMAGE [COMMAND] [ARG...]

Run a command in a new container

Options:
...............
  -i, --interactive                    Keep STDIN open even if not attached
..............
  -t, --tty                            Allocate a pseudo-TTY
............
\end{verbatim}
I have only shown the needed options. The {\em -i} keeps STDIN open, the {\em -t} assigns a pseudo-tty device to the container.

There is a docker image on Docker Hub called {\em ubuntu}.
We can get the ubuntu image and run it in a container
\begin{verbatim}
# docker run --name my_ubuntu_container -it ubuntu /bin/bash
Unable to find image 'ubuntu:latest' locally
latest: Pulling from library/ubuntu
405f018f9d1d: Downloading  3.095MB/30.42MB
405f018f9d1d: Pull complete 
Digest: sha256:b6b83d3c331794420340093eb706a6f152d9c1fa51b262d9bf34594887c2c7ac
Status: Downloaded newer image for ubuntu:latest
root@59112813a303:/# 
\end{verbatim}
 Well I get a prompt, and it looks like I am somewhere other than Void Linux. We can check with
\begin{verbatim}
root@59112813a303:/# cat usr/lib/os-release
PRETTY_NAME="Ubuntu 22.04 LTS"
NAME="Ubuntu"
VERSION_ID="22.04"
VERSION="22.04 LTS (Jammy Jellyfish)"
VERSION_CODENAME=jammy
ID=ubuntu
ID_LIKE=debian
HOME_URL="https://www.ubuntu.com/"
SUPPORT_URL="https://help.ubuntu.com/"
BUG_REPORT_URL="https://bugs.launchpad.net/ubuntu/"
PRIVACY_POLICY_URL="https://www.ubuntu.com/legal/terms-and-policies/privacy-policy"
UBUNTU_CODENAME=jammy
root@59112813a303:/# 
\end{verbatim}

Yes it looks like I am inside a container running Ubuntu. I am still root. There are no users. There is no DTE, just a command line. That is different from a VM.  
So what can we see from the Void Linux host system?
\begin{verbatim}
# docker ps
CONTAINER ID   IMAGE     COMMAND       CREATED             STATUS          PORTS     NAMES
59112813a303   ubuntu    "/bin/bash"   About an hour ago   Up 26 seconds             my_ubuntu_container

# ps -ax
  928 ?        Ss     0:00 runsv docker
  925 ?        Ss     0:00 runsv containerd
  956 ?        Sl     0:00 containerd
  958 ?        Sl     0:00 dockerd
 3632 ?        Sl     0:00 /usr/bin/containerd-shim-runc-v2 -namespace moby -id 5
\end{verbatim}
 All the daemons plus one process. 

Now lets stop the container
\begin{verbatim}
# docker stop my_ubuntu_container
my_ubuntu_container
# docker info
Client:
 Context:    default
 Debug Mode: false

Server:
 Containers: 1
  Running: 0
  Paused: 0
  Stopped: 1
 Images: 1

# docker images
REPOSITORY   TAG       IMAGE ID       CREATED       SIZE
ubuntu       latest    27941809078c   6 weeks ago   77.8MB

\end{verbatim}
Nothing is running, but we still have the ubuntu image.

So copying someone's image and running it in a container is easy.
Lets see if we can setup our own container and build some software in it.

\section{Build an application image}
Assume we have the source code for an application in some folder on our local machine. This may have been written from scratch, or we may have cloned an existing source repo. Just for a trial, I chose to use a simple C program which does some calculations related to development of wool follicles in sheep. I made a new clean directory called Folli.docker, and copied the  programs work environment into that directory
\begin{verbatim}
nevj@trinity Folli.docker$ ls
Makefile  folli.c  folli.h  folli.scr  junk
\end{verbatim}
It contains just 2 C program files (folli.c and folli.h), a Makefile, and a script (folli.scr) to do a test run. There is some other irrelevant material hidden away in a subdirectory called junk.

The task is to get this simple work environment into a docker container, with the necessary support software, such as gcc, make, C libraries, editor,... I assume that some form of cutdown Linux will have to be present in the container, to be able to use the above software.

There are two ways to proceed
\begin{itemize}
\item Use the Ubuntu container that I already have, and interactively add my little program directory and install necessary support software
\item Make a new image and container from scratch using a file called {\em Dockerfile} which specifies what to put in the new image.
\end{itemize}
Using a {\em Dockerfile} seems to be the recommended approach. The {\em Dockerfile} is placed in the top directory of the work environment which is to be made into an image - ie in my case in ~/Folli.docker. It is just a text file and its name is Dockerfile.

The official docker guide to writing Dockerfiles is here~\cite{dock:03}. A better explanatory document is here~\cite{dock:04} or here~\cite{dock:05} or here~\cite{dock:06}.
After reading all those guides, I am still not clear on some issues. The only way is to start making a Dockerfile and learn from mistakes. 

The first thing is to specify the parent image - that is the cutdown Linux that is to support my work environment. In Dockerfile that is done with a FROM command.For example
\begin{verbatim}
FROM ubuntu:18.04
\end{verbatim}
so you can specify the version, as well as the distro name.

 I would rather use Alpine than Ubuntu, so what I want to tell my Dockerfile to do is to fetch from docker-hub a cut down Alpine image, and add to it my local working directory to make a custom image. That is easy. We write a simple Dockerfile
\begin{verbatim}
FROM alpine
COPY . .
\end{verbatim}
The COPY statement says copy everything from current directory (first '.') in host system ( ie where the Dockerfile is) to '.' in the custom built image.
I also added a {\em .dockerignore} file to the ~/Filli.docker directory, so that the {\em junk} directory will not be included in the image.

Then we use these Dockerfile instructions to build a custom image, working from witin the ~/Folli.docker directory
\begin{verbatim}
# docker build .
Sending build context to Docker daemon  15.36kB
Step 1/2 : FROM alpine
latest: Pulling from library/alpine
530afca65e2e: Pull complete 
Digest: sha256:7580ece7963bfa863801466c0a488f11c86f85d9988051a9f9c68cb27f6b7872
Status: Downloaded newer image for alpine:latest
 ---> d7d3d98c851f
Step 2/2 : COPY . .
 ---> 9fbddcffedb9
Successfully built 9fbddcffedb9
# 
\end{verbatim}
We can see the custom image we built with
\begin{verbatim}
# docker images
REPOSITORY                            TAG       IMAGE ID       CREATED          SIZE
<none>                                <none>    9fbddcffedb9   12 minutes ago   5.54MB
alpine                                latest    d7d3d98c851f   5 days ago       5.53MB
ubuntu                                latest    27941809078c   6 weeks ago      77.8MB
\end{verbatim}

So it called it $<none>$ which is not ideal, it needs a custom name, we will fix that later. The other wo images ( called 'ubuntu' and 'alpine') are the parent images of those distros, as downloaded from DockerHub.

Now we can run the image $<none>$ as an interactive container with 
\begin{verbatim}
# docker run --name my_custom_container -it 9fbddcffedb9 /bin/ash
/ # 
\end{verbatim}
Well that is fairly brief. Lets see what is running, from a host system window do
\begin{verbatim}
# docker stats
CONTAINER ID   NAME                  CPU %     MEM USAGE / LIMIT   MEM %     NET I/O         BLOCK I/O   PIDS
db6329f387a8   my_custom_container   0.00%     676KiB / 62.79GiB   0.00%     3.01kB / 522B   0B / 0B     1
\end{verbatim}
So a container is running , and we succeeded in giving it a name.. Lets go to the interactive prompt 9ie the container window) and see what it contains
\begin{verbatim}
/ # pwd
/
/ # ls -aF
./             bin/           home/          root/          usr/
../            dev/           lib/           run/           var/
.dockerenv*    etc/           media/         sbin/
.dockerignore  folli.c        mnt/           srv/
Dockerfile     folli.h        opt/           sys/
Makefile       folli.scr      proc/          tmp/
/ # 
\end{verbatim}
 So my workfiles are there, but it put it all in the root directory.  That is not very nice either.. more things to fix later.

Now can I build my little software collection with make?
\begin{verbatim}
/ # make
/bin/ash: make: not found
/ # 
\end{verbatim}
 No. There is no {\em make} command. Not surprising , the Dockerfile did not add make and other build requirements to the alpine image. 

Lets see if we can add it interactively (just to learn what is needed)
\begin{verbatim}
/ # apk add make
(1/1) Installing make (4.3-r0)
Executing busybox-1.35.0-r15.trigger
OK: 6 MiB in 15 packages
/ # which make
/usr/bin/make
/ # 
\end{verbatim}
And we had better have a few other build requirements
\begin{verbatim}
/ # apk add gcc
(1/10) Installing libgcc (11.2.1_git20220219-r2)
(2/10) Installing libstdc++ (11.2.1_git20220219-r2)
(3/10) Installing binutils (2.38-r3)
(4/10) Installing libgomp (11.2.1_git20220219-r2)
(5/10) Installing libatomic (11.2.1_git20220219-r2)
(6/10) Installing gmp (6.2.1-r2)
(7/10) Installing isl22 (0.22-r0)
(8/10) Installing mpfr4 (4.1.0-r0)
(9/10) Installing mpc1 (1.2.1-r0)
(10/10) Installing gcc (11.2.1_git20220219-r2)
Executing busybox-1.35.0-r15.trigger
OK: 109 MiB in 25 packages
/ # 
\end{verbatim}
That seems to have dragged in all the necessary tools.
It has {\em vi} and {\em more} , so lets try a compile again.
\begin{verbatim}
/ # make
cc -v  -g  -static   -c -o folli.o folli.c
Using built-in specs.
COLLECT_GCC=cc
Target: x86_64-alpine-linux-musl
Configured with: /home/buildozer/aports/main/gcc/src/gcc-11.2.1_git20220219/configure --prefix=/usr --mandir=/usr/share/man --infodir=/usr/share/info --build=x86_64-alpine-linux-musl --host=x86_64-alpine-linux-musl --target=x86_64-alpine-linux-musl --with-pkgversion='Alpine 11.2.1_git20220219' --enable-checking=release --disable-fixed-point --disable-libstdcxx-pch --disable-multilib --disable-nls --disable-werror --disable-symvers --enable-__cxa_atexit --enable-default-pie --enable-default-ssp --enable-cloog-backend --enable-languages=c,c++,d,objc,go,fortran,ada,jit --disable-libssp --disable-libmpx --disable-libmudflap --disable-libsanitizer --enable-shared --enable-threads --enable-tls --enable-host-shared --with-system-zlib --with-linker-hash-style=gnu
Thread model: posix
Supported LTO compression algorithms: zlib
gcc version 11.2.1 20220219 (Alpine 11.2.1_git20220219) 
COLLECT_GCC_OPTIONS='-v' '-g' '-static' '-c' '-o' 'folli.o' '-mtune=generic' '-march=x86-64'
 /usr/libexec/gcc/x86_64-alpine-linux-musl/11.2.1/cc1 -quiet -v folli.c -quiet -dumpbase folli.c -dumpbase-ext .c -mtune=generic -march=x86-64 -g -version -o /tmp/cchpbCjP.s
GNU C17 (Alpine 11.2.1_git20220219) version 11.2.1 20220219 (x86_64-alpine-linux-musl)
	compiled by GNU C version 11.2.1 20220219, GMP version 6.2.1, MPFR version 4.1.0, MPC version 1.2.1, isl version isl-0.22-GMP

GGC heuristics: --param ggc-min-expand=100 --param ggc-min-heapsize=131072
ignoring nonexistent directory "/usr/local/include"
ignoring nonexistent directory "/usr/lib/gcc/x86_64-alpine-linux-musl/11.2.1/../../../../x86_64-alpine-linux-musl/include"
ignoring nonexistent directory "/usr/include/fortify"
#include "..." search starts here:
#include <...> search starts here:
 /usr/include
 /usr/lib/gcc/x86_64-alpine-linux-musl/11.2.1/include
End of search list.
GNU C17 (Alpine 11.2.1_git20220219) version 11.2.1 20220219 (x86_64-alpine-linux-musl)
	compiled by GNU C version 11.2.1 20220219, GMP version 6.2.1, MPFR version 4.1.0, MPC version 1.2.1, isl version isl-0.22-GMP

GGC heuristics: --param ggc-min-expand=100 --param ggc-min-heapsize=131072
Compiler executable checksum: 032e78b3e0ace96e0ed58573fd512cc9
folli.c:12:17: fatal error: stdio.h: No such file or directory
   12 | #include        <stdio.h>
      |                 ^~~~~~~~~
compilation terminated.
make: *** [<builtin>: folli.o] Error 1
/ # 
\end{verbatim}

 Not quite, we forgot the include files for standard C.
\begin{verbatim}
/ # apk add libc-dev
(1/2) Installing musl-dev (1.2.3-r0)
(2/2) Installing libc-dev (0.7.2-r3)
OK: 119 MiB in 27 packages
/ # 
\end{verbatim}
It seem libc-dev is a metapackage that pulls in musl-dev. So Alpine uses musl not glibc. OK, it should still work.
Try again
\begin{verbatim}
/ # make
cc -v  -g  -static   -c -o folli.o folli.c
Using built-in specs.
COLLECT_GCC=cc
Target: x86_64-alpine-linux-musl
Configured with: /home/buildozer/aports/main/gcc/src/gcc-11.2.1_git20220219/configure --prefix=/usr --mandir=/usr/share/man --infodir=/usr/share/info --build=x86_64-alpine-linux-musl --host=x86_64-alpine-linux-musl --target=x86_64-alpine-linux-musl --with-pkgversion='Alpine 11.2.1_git20220219' --enable-checking=release --disable-fixed-point --disable-libstdcxx-pch --disable-multilib --disable-nls --disable-werror --disable-symvers --enable-__cxa_atexit --enable-default-pie --enable-default-ssp --enable-cloog-backend --enable-languages=c,c++,d,objc,go,fortran,ada,jit --disable-libssp --disable-libmpx --disable-libmudflap --disable-libsanitizer --enable-shared --enable-threads --enable-tls --enable-host-shared --with-system-zlib --with-linker-hash-style=gnu
Thread model: posix
Supported LTO compression algorithms: zlib
gcc version 11.2.1 20220219 (Alpine 11.2.1_git20220219) 
COLLECT_GCC_OPTIONS='-v' '-g' '-static' '-c' '-o' 'folli.o' '-mtune=generic' '-march=x86-64'
 /usr/libexec/gcc/x86_64-alpine-linux-musl/11.2.1/cc1 -quiet -v folli.c -quiet -dumpbase folli.c -dumpbase-ext .c -mtune=generic -march=x86-64 -g -version -o /tmp/ccPGBkNe.s
GNU C17 (Alpine 11.2.1_git20220219) version 11.2.1 20220219 (x86_64-alpine-linux-musl)
	compiled by GNU C version 11.2.1 20220219, GMP version 6.2.1, MPFR version 4.1.0, MPC version 1.2.1, isl version isl-0.22-GMP

GGC heuristics: --param ggc-min-expand=100 --param ggc-min-heapsize=131072
ignoring nonexistent directory "/usr/local/include"
ignoring nonexistent directory "/usr/lib/gcc/x86_64-alpine-linux-musl/11.2.1/../../../../x86_64-alpine-linux-musl/include"
ignoring nonexistent directory "/usr/include/fortify"
#include "..." search starts here:
#include <...> search starts here:
 /usr/include
 /usr/lib/gcc/x86_64-alpine-linux-musl/11.2.1/include
End of search list.
GNU C17 (Alpine 11.2.1_git20220219) version 11.2.1 20220219 (x86_64-alpine-linux-musl)
	compiled by GNU C version 11.2.1 20220219, GMP version 6.2.1, MPFR version 4.1.0, MPC version 1.2.1, isl version isl-0.22-GMP

GGC heuristics: --param ggc-min-expand=100 --param ggc-min-heapsize=131072
Compiler executable checksum: 032e78b3e0ace96e0ed58573fd512cc9
folli.c:23:1: warning: return type defaults to 'int' [-Wimplicit-int]
   23 | main(int argc,char *argv[])
      | ^~~~
COLLECT_GCC_OPTIONS='-v' '-g' '-static' '-c' '-o' 'folli.o' '-mtune=generic' '-march=x86-64'
 /usr/lib/gcc/x86_64-alpine-linux-musl/11.2.1/../../../../x86_64-alpine-linux-musl/bin/as -v --gdwarf-5 --64 -o folli.o /tmp/ccPGBkNe.s
GNU assembler version 2.38 (x86_64-alpine-linux-musl) using BFD version (GNU Binutils) 2.38
COMPILER_PATH=/usr/libexec/gcc/x86_64-alpine-linux-musl/11.2.1/:/usr/libexec/gcc/x86_64-alpine-linux-musl/11.2.1/:/usr/libexec/gcc/x86_64-alpine-linux-musl/:/usr/lib/gcc/x86_64-alpine-linux-musl/11.2.1/:/usr/lib/gcc/x86_64-alpine-linux-musl/:/usr/lib/gcc/x86_64-alpine-linux-musl/11.2.1/../../../../x86_64-alpine-linux-musl/bin/
LIBRARY_PATH=/usr/lib/gcc/x86_64-alpine-linux-musl/11.2.1/:/usr/lib/gcc/x86_64-alpine-linux-musl/11.2.1/../../../../x86_64-alpine-linux-musl/lib/../lib/:/usr/lib/gcc/x86_64-alpine-linux-musl/11.2.1/../../../../lib/:/lib/../lib/:/usr/lib/../lib/:/usr/lib/gcc/x86_64-alpine-linux-musl/11.2.1/../../../../x86_64-alpine-linux-musl/lib/:/usr/lib/gcc/x86_64-alpine-linux-musl/11.2.1/../../../:/lib/:/usr/lib/
COLLECT_GCC_OPTIONS='-v' '-g' '-static' '-c' '-o' 'folli.o' '-mtune=generic' '-march=x86-64' '-dumpdir' 'folli.'
gcc  -o folli  folli.o  -lm   
/ # 

/ # ls -aF
./             bin/           folli.o        opt/           sys/
../            dev/           folli.scr      proc/          tmp/
.dockerenv*    etc/           home/          root/          usr/
.dockerignore  folli*         lib/           run/           var/
Dockerfile     folli.c        media/         sbin/
Makefile       folli.h        mnt/           srv/
/ # 
\end{verbatim}
 So it worked. Now lets try and run the binary, which is called folli. We will use the script folli.scr. Alpine has {\em sh} shell, so it should work
\begin{verbatim}
/ # cat folli.scr
./folli <<eoi
2000000
3.0
4.303449
0.000002017
450.
64
86
1.0e8
35. 33. 30.
eoi

Now run the script
/ # sh -ex folli.scr
+ ./folli
 number of primary sites 2000000
 So/P ratio   3.00
 growthrate - slope of log_wt/log_age line  4.30345
 growth intercept - of log_wt/log_age line 0.0000020170
 follicle initiation rate - increase per timeincrement per cm sq 450.0000
 time of start of primary follicle initiation period       64
 time of start of secondary original follicle initiation period       86
 number of founder cells at time zero 100000000.0
 average number of cells per p,so,sd follicle  35.000   33.000   30.000 


 time     foll  pfoll   sofoll   sdfoll diffoundcel  foundcel  wght surfarea folirate celbrate    d    pd   sod   sdd    sd  rat
between P and So periods
 61         0        0         0         0           0   107054834     97.2    190.3    85633  0.07055      0     0      0      0      0   0.0
.......
  out  of founder cells 
300  27069651  2171768   6543389  18354494   842578546           0  92229.7  18372.0  8267400  0.01434   1473   118    356    999   1355  11.5
  adult S+P density per cm sq =     2572
 adult P density per cm sq =       206
 adult So density per cm sq =      622
 adult Sd density per cm sq =     1744
 adult S density per cm sq =     2365
 adult S/P ratio =     11.5
 p interval = 14
 so interval = 18
/ # 
\end{verbatim}
 So the script runs. Everything is OK, just terribly untidy.
So lets quit and cleanup
\begin{verbatim}
/ # exit
# 
\end{verbatim}
 So we exit back to the Void host system

\subsection{Saving containers}
Before we get rid of the $<none>$ image , I need to see whether the stuff I added interactively while I had the container running was preserved when I killed the running container with 
\begin{verbatim}
# docker container prune
\end{verbatim}
So lets run $<none>$ again in a new container
\begin{verbatim}
#docker run --name my_second_container -it 9fbddcffedb9 /bin/ash

/ # ls -aF
./             bin/           home/          root/          usr/
../            dev/           lib/           run/           var/
.dockerenv*    etc/           media/         sbin/
.dockerignore  folli.c        mnt/           srv/
Dockerfile     folli.h        opt/           sys/
Makefile       folli.scr      proc/          tmp/
/ # which make
/ # 
\end{verbatim}
 So , no , the stuff I added is not saved in the $<none>$ image.
It contains only the stuff put there by Dockerfile. 
Lesson learnt!  Dont prune a container without saving it.
How to save a container?
This seems likely
\begin{verbatim}
# docker --help
.....
  commit      Create a new image from a container's changes
......
# docker commit --help

Usage:  docker commit [OPTIONS] CONTAINER [REPOSITORY[:TAG]]

Create a new image from a container's changes

Options:
  -a, --author string    Author (e.g., "John Hannibal Smith
                         <hannibal@a-team.com>")
  -c, --change list      Apply Dockerfile instruction to the created image
  -m, --message string   Commit message
  -p, --pause            Pause container during commit (default true)
# 
\end{verbatim}
Well, thats all very well, but how do I give it a name? Lets get an example
\begin{verbatim}
docker commit c3f279d17e0a  svendowideit/testimage:version3
\end{verbatim}
That came from the docker documentation. It looks like convention is to use a repository name ( like nevj/dockerfiletest) and a tag indicating version (like :version1). So lets reinstall the software (make,gcc,libc-dev) and run the make, so I can tell it from the $<none>$ image. Then try to save the container
\begin{verbatim}
# docker ps
CONTAINER ID   IMAGE          COMMAND      CREATED          STATUS          PORTS     NAMES
c8a4532843d8   9fbddcffedb9   "/bin/ash"   32 minutes ago   Up 32 minutes             my_second_container
# 

# docker commit c8a4532843d8 nevj/dockerfiletest:version1
sha256:d15df250f7ac209b987c1daf264d773f4954411324989b470119920374589216
# 
# docker images
REPOSITORY            TAG        IMAGE ID       CREATED         SIZE
nevj/dockerfiletest   version1   d15df250f7ac   9 seconds ago   124MB
alpine                latest     d7d3d98c851f   6 days ago      5.53MB
ubuntu                latest     27941809078c   6 weeks ago     77.8MB
# 
\end{verbatim}
 Yes looks like we have it saved as an image.
Now what happens if I exit from the running container?
\begin{verbatim}
In the container window
/ # exit
# 

In the host system window
# docker ps
CONTAINER ID   IMAGE     COMMAND   CREATED   STATUS    PORTS     NAMES
# docker images
REPOSITORY            TAG        IMAGE ID       CREATED          SIZE
nevj/dockerfiletest   version1   d15df250f7ac   12 minutes ago   124MB
alpine                latest     d7d3d98c851f   6 days ago       5.53MB
ubuntu                latest     27941809078c   6 weeks ago      77.8MB
\end{verbatim}
 There is no running container now, but the image is still saved. 
\begin{verbatim}
# docker info
.....
Server:
 Containers: 2
  Running: 0
  Paused: 0
  Stopped: 2
 Images: 4
\end{verbatim}
I still have 2 containers, both are stopped, and there are 4 images although {\em docker images} only lists 3 images?
So I guess we still have an unsaved container, and its image, still hanging around?  Will it survive a reboot?

\begin{verbatim}
After reboot

# docker info
Server:
 Containers: 2
  Running: 0
  Paused: 0
  Stopped: 2
 Images: 4
\end{verbatim}
Nothing has changed. Even stopped containers and their images survive a reboot.


\subsection{Big cleanup}
w do I get rid of the stopped container without interfering with the saved image nevj/dockerfiletest?
\begin{verbatim}
# docker rm --help

Usage:  docker rm [OPTIONS] CONTAINER [CONTAINER...]

Remove one or more containers

\end{verbatim}
 But I cant remember its name? 
So I am forced to use prune again
\begin{verbatim}
# docker container prune --help

Usage:  docker container prune [OPTIONS]

Remove all stopped containers
\end{verbatim}
 That should do it
\begin{verbatim}
# docker container prune
WARNING! This will remove all stopped containers.
Are you sure you want to continue? [y/N] y
Deleted Containers:
c8a4532843d8a911c2e43249f8c24ee87a0562bd3114598413707c243bc3ecac
db6329f387a885c012827a0ff80403fd2279a228530f317e375d3b7b2c6bab24

Total reclaimed space: 236.5MB
# 
# docker images
REPOSITORY            TAG        IMAGE ID       CREATED          SIZE
nevj/dockerfiletest   version1   d15df250f7ac   37 minutes ago   124MB
alpine                latest     d7d3d98c851f   6 days ago       5.53MB
ubuntu                latest     27941809078c   6 weeks ago      77.8MB
# 
\end{verbatim}
That worked , and the 3 images are still saved, but
\begin{verbatim}
# docker info
....
Server:
 Containers: 0
  Running: 0
  Paused: 0
  Stopped: 0
 Images: 4
\end{verbatim}.
The container is gone, but why 4 images, when {\em docker images} only lists 3 images?
Tried {\em docker image prune} but {\em docker info} still lists 4 images. 
I give up. I think docker has lost count!

Now, if I run that saved nevj/dockerfiletest image in another new container, does it contain all the interactive mods?
\begin{verbatim}
# docker run --name my_third_container -it d15df250f7ac /bin/ash
/ # 

/ # ls -aF
./             bin/           folli.o        opt/           sys/
../            dev/           folli.scr      proc/          tmp/
.dockerenv*    etc/           home/          root/          usr/
.dockerignore  folli*         lib/           run/           var/
Dockerfile     folli.c        media/         sbin/
Makefile       folli.h        mnt/           srv/
/ # which make
/usr/bin/make
/ # 
\end{verbatim}
 It is all there. So the save was successful.
This time lets see if we can kiil the container while it is running
\begin{verbatim}
# docker kill my_third_container
my_third_container
# docker ps
CONTAINER ID   IMAGE     COMMAND   CREATED   STATUS    PORTS     NAMES
# docker images
REPOSITORY            TAG        IMAGE ID       CREATED          SIZE
nevj/dockerfiletest   version1   d15df250f7ac   54 minutes ago   124MB
alpine                latest     d7d3d98c851f   6 days ago       5.53MB
ubuntu                latest     27941809078c   6 weeks ago      77.8MB
# 
\end{verbatim}
It works, and the prompt returns in the container window.
But
\begin{verbatim}
# docker info
.....
Server:
 Containers: 1
  Running: 0
  Paused: 0
  Stopped: 1
 Images: 4
\end{verbatim}
So kill only stopped the container, it did not remove it. So
\begin{verbatim}
# docker rm my_third_container
my_third_container
# docker info
....
Server:
 Containers: 0
  Running: 0
  Paused: 0
  Stopped: 0
 Images: 4
\end{verbatim}
 Yes {\em docker rm} works like {\em docker container prune} but {\em docker kill} only stops the container.

At last, I think I understand container management.

\subsection{Redo the Dockerfile}

\section{Discussion}



\begin{thebibliography}{99}

\bibitem{guru:99}
Docker tutorial.
URL https://www.guru99.com/docker-tutorial.html

\bibitem{dock:00}
Docker get started 
URL https://docs.docker.com/get-started/

\bibitem{dock:01}
Docker Desktop
URL https://docs.docker.com/desktop/install/linux-install/

\bibitem{dock:02}
Docker Hub
URL https://hub.docker.com/

\bibitem{dock:03}
Official Dockerfile documnet
URL https://docs.docker.com/develop/develop-images/dockerfile\_best-practices/

\bibitem{dock:04}
Dockerfile Guide
URL https://medium.com/@BeNitinAgarwal/best-practices-for-working-with-dockerfiles-fb2d22b78186

\bibitem{dock:05}
Dockerfile Guide
URL https://thenewstack.io/docker-basics-how-to-use-dockerfiles/

\bibitem{dock:06}
Dockerfile Guide
URL https://jfrog.com/knowledge-base/a-beginners-guide-to-understanding-and-building-docker-images/

\bibitem{void:01}
Void Linux Docker Images
URL https://github.com/void-linux/void-docker

\bibitem{foss:21}
FOSS article on Project Trident
URL https://news.itsfoss.com/project-trident-discontinues/


\bibitem{libr:22}
LibreWolf source code website.
URL https://gitlab.com/librewolf-community/browser/source


\bibitem{wate:22} 
Waterfox website. URL https://www.waterfox.net


\end{thebibliography}
\end{document}
