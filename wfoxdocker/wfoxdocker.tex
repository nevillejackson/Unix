%
% Draft  document voiddocker.tex
%
 
\documentclass{article}  % Latex2e
\usepackage{graphicx,lscape,subfigure}
 

\title{Making a docker image to run the Waterfox browser}
\author{Neville Jackson}
\date{4 Sep 2022} 

\begin{document} 

\maketitle      

\section{Introduction} 
This project is about learning how to compose a Dockerfile which runs an X11 application. The Waterfox browser was chosen as an example X11 application, because Waterfox is not available as a package in most Linux distributions, so the result might actually be a useful means of obtaining Waterfox without having to download a tarfile and do a local install. 

Getting a running Docker container to talk to the X11 server in the host system is relatively easy. Waterfox is a large and complex X application, and most of the challenge of this project is in setting up in the Docker image a working environmejnt for Waterfox. 

\section{What is involved in a normal Waterfox install?}
There is a document~\cite{wate:22a} on that.  Briefly the steps are
\begin{enumerate}
\item Download a tarfile from the Waterfox website~\cite{wate:22} or from the Github site~\cite{wate:22b}.
\item  Unpack the tarfile in /usr/local/src
\begin{verbatim}
bzip2 -dc waterfox-G4.1.1.1.en-US.linux-x86_64.tar.bz2 | tar xvf -
\end{verbatim}
\item Make a link in /usr/local/bin to point to the Waterfox binary
\begin{verbatim}
ln /usr/local/src/waterfox/waterfox /usr/local/bin/waterfox
\end{verbatim}
\item Define the desktop icon by adding a file waterfox.desktop to the directory /usr/share applications. The file waterfox.desktop does not come with the tarfile, but is available here~\cite{wate:22a}
\end{enumerate}
We need to implement steps 1 to 3 in a Dockerfile. Step 4 can be omitted, as there will be no icon if running Waterfox in a container. The container will have to be started from the command line.


\section{First attempt at a Dockerfile, using  Debian parent image}
I chose Debian parent image for a first attempt, because Debian is familiar and is most likely to compatable with Waterfox. 

\begin{thebibliography}{99}

\bibitem{guru:99}
Docker tutorial.
URL https://www.guru99.com/docker-tutorial.html

\bibitem{dock:00}
Docker get started
URL https://docs.docker.com/get-started/

\bibitem{dock:01}
Docker Desktop
URL https://docs.docker.com/desktop/install/linux-install/

\bibitem{dock:02}
Docker Hub
URL https://hub.docker.com/

\bibitem{dock:03}
Official Dockerfile documnet
URL https://docs.docker.com/develop/develop-images/dockerfile\_best-practices/


\bibitem{dock:04}
Dockerfile Guide
URL https://medium.com/@BeNitinAgarwal/best-practices-for-working-with-dockerfil
es-fb2d22b78186

\bibitem{dock:05}
Docker Basics: How to use Dockerfiles
URL https://thenewstack.io/docker-basics-how-to-use-dockerfiles/

\bibitem{dock:06}
A Beginners Guide to Understanding and Building Docker Images
URL https://jfrog.com/knowledge-base/a-beginners-guide-to-understanding-and-building-docker-images/

\bibitem{dock:07}
Best practices for writing Dockerfiles
URL https://docs.docker.com/develop/develop-images/dockerfile\_best-practices/


\bibitem{dock:09}
Creating a Docke Image for your Application
URL https://www.stereolabs.com/docs/docker/creating-your-image/

\bibitem{dock:10}
Docker Containerization Cookbook” - Hot Recipes for Docker Automation
URL https://distrowatch.tradepub.com/free/w\_java39/prgm.cgi?a=1

\bibitem{void:01}
Void Linux Docker Images
URL https://github.com/void-linux/void-docker


\bibitem{libr:22}
LibreWolf source code website.
URL https://gitlab.com/librewolf-community/browser/source


\bibitem{wate:22} 
Waterfox website. URL https://www.waterfox.net

\bibitem{wate:22a}
Install the Waterfox browser on a Linux system. URL https://github.com/nevillejackson/Unix/blob/main/waterfox/waterfox.pdf

\bibitem{wate:22b}
WaterfocCo Github site URL https://github.com/WaterfoxCo/Waterfox/releases

\end{thebibliography}
\end{document}
